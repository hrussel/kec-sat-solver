\documentclass[12pt,lettersize]{article}
\usepackage[spanish]{babel}
\usepackage[utf8]{inputenc}
\usepackage{listings}
\lstset{
  language=C
}
\begin{document}

\section{Introducción}

\subsection{Motivación del proyecto}

\subsection{Breve descripción del problema} 
Si $S=\{x_1,x_2,\ldots,x_n\}$ es un conjunto finito de variables booleanas,
llamemos $\bar{S}=\{\bar{x_1},\bar{x_2},\ldots,\bar{x_n}\}$ al conjunto formado
por la negación de las variables en $S$. 


Una \emph{cláusula} booleana

El \emph{problema de la satisfacción booleana (SAT)} consiste en lo siguiente:
dados un conjunto finito de variables booleanas $x_1,x_2,\ldots,x_n$ y una
fórmula booleana $E(x_1,x_2,\ldots,x_n,\bar{x_1},\bar{x_2},\ldots,\bar{x_n})$ en
forma normal conjuntiva (CNF), determinar si existe una asignación de valores de
verdad a las variables tal que la fórmula sea verdad.

\section{Diseño}

\section{Detalles de implementación}


\subsection{Problemas encontrados y la manera como fueron resueltos}


\begin{texttt}
d
\end{texttt}

\section{Instrucciones de operación}
Para emplear la aplicación, escribir en la consola el comando
\\ \texttt{kecss -f filename}
\section{Estado Actual}

\section{Conclusiones y recomendaciones}


\end{document}
