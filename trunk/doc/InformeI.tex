\documentclass[12pt,lettersize]{article}
\usepackage[spanish]{babel}
\usepackage[utf8]{inputenc}
\usepackage{listings}
\lstset{
  language=C,
  basicstyle=\small
}
\begin{document}
\setlength{\parskip}{2.5mm}
\section{Introducción}

\subsection{Motivación del proyecto}
En 1971, Stephen~Cook propuso en su trabajo una nueva categoría de complejidad
de problemas de decisión computacionales, a la que llamó problemas
\emph{NP-completos}. La caracterización de esta categoría se hace sobre estas
dos propiedades:
\begin{itemize}
  \setlength{\itemsep}{0ex }
  \item Todos los problemas \emph{NP-completos} pueden ser verificados en tiempo
    $O(p(n))$, donde $p(n)$ es un polinomio en función de $n$ el tamaño de la
    instancia del problema. 
  \item Todos los problemas en \emph{NP} pueden ser reducidos en tiempo
    $O(p(n))$ a algún problema \emph{NP-completo}, donde $p(n)$ es un polinomio
    en función de $n$ el tamaño de la instancia del problema que es reducido.
\end{itemize}

Ahora bien, fueron Cook y Leonid~Levin quienes encontraron, de forma
independiente, el primer problema en esta categoría \emph{NP-completos}: el
problema de la \emph{satisfacción booleana~(SAT)}. Un año después, Richard~Karp
identificó otros 21 problemas en esta categoría, los cuales tenían la notoria
característica de que para ellos no se conoce un algoritmo polinomial (en
función del tamaño de la instancia) que les de solución, una cualidad que
comparten todos los problemas en esta clase, junto al hecho de que todos estos
problemas ocurren con una marcada frecuencia en el área de la computación. Sin
embargo, la característica más especial de todos estos problemas es el segundo
ítem de arriba: encontrar un algoritmo polinomial para tan sólo uno de estos
problemas es encontrar un algoritmo polinomial para todos.

De modo pues que la motivación para este proyecto estriba en el hecho de que
\emph{SAT} fue el primer problema que se encontró en \emph{NP-Completos} y que
todos los problemas en esta clase son reducibles en tiempo polinomial a
él. Siendo así y bajo el supuesto de que estas reducciones a \emph{SAT} se
caractericen por polinomios de bajo grado y coeficientes pequeños, cualquier
mejora en tiempo que se pueda realizar a los algoritmos exponenciales hoy
conocidos para resolver el problema \emph{SAT} es una mejora para los algoritmos
exponenciales conocidos para los demás problemas en \emph{NP-completos}.

\subsection{Breve descripción del problema} 
Si $S=\{x_1,x_2,\ldots,x_n\}$ es un conjunto finito de variables booleanas,
llamemos $\bar{S}=\{\bar{x_1},\bar{x_2},\ldots,\bar{x_n}\}$ al conjunto formado
por la negación de las variables en $S$. 


El \emph{problema de la satisfacción booleana~(SAT)} consiste en lo siguiente:
dados un conjunto finito de variables booleanas $x_1,x_2,\ldots,x_n$ y una
fórmula booleana $E(x_1,x_2,\ldots,x_n,\bar{x_1},\bar{x_2},\ldots,\bar{x_n})$ en
forma normal conjuntiva (CNF), determinar si existe una asignación de valores de
verdad a las variables tal que la fórmula sea verdad.

\section{Diseño}

\section{Detalles de implementación}

\begin{lstlisting}
typedef struct SAT_status{    
    int num_vars;
    int num_clauses;
    clause *formula;
    list *pos_watched_list;
    list *neg_watched_list;
    stack backtracking_status;
    int *model;                     
} SAT_status;
\end{lstlisting}


\begin{lstlisting}
typedef struct clause{
    int size;
    variable* head_watcher;
    variable* tail_watcher;
    variable* literals;
} clause;
\end{lstlisting}


\subsection{Problemas encontrados y la manera como fueron resueltos}


\begin{texttt}
d
\end{texttt}

\section{Instrucciones de operación}
Para emplear la aplicación, escribir en la consola el comando
\\ \texttt{sat -f inputfilename -o outputfilename}
\section{Estado Actual}

\section{Conclusiones y recomendaciones}


\end{document}
