\documentclass[12pt,lettersize]{article}
\usepackage[spanish]{babel}
\usepackage[utf8]{inputenc}
\usepackage{amsmath}
\usepackage{amssymb}
\usepackage{listings}
\lstset{
  language=C,
  basicstyle=\small
}
\begin{document}
\setlength{\parskip}{2.5mm}
\section{Introducción}

\subsection{Motivación del proyecto}
En 1971, Stephen~Cook propuso en su trabajo una nueva categoría de complejidad
de problemas de decisión computacionales, a la que llamó problemas
\emph{NP-completos}. La caracterización de esta categoría se hace sobre estas
dos propiedades:
\begin{itemize}
  \setlength{\itemsep}{0ex }
  \item Todos los problemas \emph{NP-completos} pueden ser verificados en tiempo
    $O(p(n))$, donde $p(n)$ es un polinomio en función de $n$ el tamaño de la
    instancia del problema. 
  \item Todos los problemas en \emph{NP} pueden ser reducidos en tiempo
    $O(p(n))$ a algún problema \emph{NP-completo}, donde $p(n)$ es un polinomio
    en función de $n$ el tamaño de la instancia del problema que es reducido.
\end{itemize}

Ahora bien, fueron Cook y Leonid~Levin quienes encontraron, de forma
independiente, el primer problema en esta categoría \emph{NP-completos}: el
problema de la \emph{satisfacción booleana~(SAT)}. Un año después, Richard~Karp
identificó otros 21 problemas en esta categoría, los cuales tenían la notoria
característica de que para ellos no se conoce un algoritmo polinomial (en
función del tamaño de la instancia) que les de solución, una cualidad que
comparten todos los problemas en esta clase, junto al hecho de que todos estos
problemas ocurren con una marcada frecuencia en el área de la computación. Sin
embargo, la característica más especial de todos estos problemas es el segundo
ítem de arriba: encontrar un algoritmo polinomial para tan sólo uno de estos
problemas es encontrar un algoritmo polinomial para todos.

De modo pues que la motivación para este proyecto estriba en el hecho de que
\emph{SAT} fue el primer problema que se encontró en \emph{NP-Completos} y que
todos los problemas en esta clase son reducibles en tiempo polinomial a
él. Siendo así y bajo el supuesto de que estas reducciones a \emph{SAT} se
caractericen por polinomios de bajo grado y coeficientes pequeños, cualquier
mejora en tiempo que se pueda realizar a los algoritmos exponenciales hoy
conocidos para resolver el problema \emph{SAT} es una mejora para los algoritmos
exponenciales conocidos para los demás problemas en \emph{NP-completos}.

\subsection{Breve descripción del problema} 
Si $S=\{x_1,x_2,\ldots,x_n\}$ es un conjunto finito de variables booleanas,
llamemos $\bar{S}=\{\bar{x_1},\bar{x_2},\ldots,\bar{x_n}\}$ al conjunto formado
por la negación de las variables en $S$. 

El \emph{problema de la satisfacción booleana~(SAT)} consiste de la forma
general de las instancias al problema y de la pregunta:
\vspace{-2.5mm}
\begin{enumerate}
\item la forma general de las instancias: Dados un conjunto finito de variables
  booleanas $x_1,x_2,\ldots,x_n$ y una fórmula booleana $F(x_1,x_2,\ldots,x_n)$
  en forma normal conjuntiva (CNF).
\item la pregunta cuya respuesta se quiere determinar: ¿existe una asignación de
  valores de verdad a las variables $x_1,\ldots, x_n$ tal que la fórmula sea
  verdad?
\end{enumerate}

\section{Diseño}

\subsection{Fase de selección de variables a asignar}

\subsection{Fase de Deducción}

\subsubsection{Propagación de cláusulas unitarias}

\subsubsection{Eliminación de literales puros}


\section{Detalles de implementación}

El proceso

La implementación de la propagación de cláusulas unitarias con 2 testigos por
cláusula (\cite{} los llama \emph{2-watched literals}) empleada por
\emph{zChaff} asocia a cada literal $x_i, i\in{1,\ldots,n}$ un par de listas, la
primera de ellas tiene como elementos a todas las cláusulas en las que la el
literal $x_i$ ocurre no negado (polaridad positiva) como testigo o \emph{watched
  literal}. La segunda lista asociada a $x_i$ tiene como elementos a todas las
cláusulas en las que el literal $\overline{x_i}$ ocurre como testigo o
\emph{watched literal}. 

En la implementación que aquí se describe, se optó por por los campos {\tt
  pos\_watched\_list} y {\tt neg\_watched\_list} en el tipo {\tt
  SAT\_status}. Cada uno de éstos es un arreglo de cabezas de listas, de forma
que {\tt pos\_watched\_list[i]} sea la cabeza de la lista cuyos elementos son
las cláusulas en las que el literal $x_i$ ocurre como
\emph{watcher}. Análogamente ocurre con {\tt neg\_watched\_list[i]}.

El campo {\tt model} del tipo {\tt SAT\_status} es un arreglo de enteros tal que
{\tt model[i]} es el valor de asignación que se prueba para la variable
$x_i$. El {\tt model} indica cuál nodo de la arborescencia del
\emph{backtracking} se está considerando en un determinado instante de la
ejecución\footnote{Véase la sección \ref{backtracking} que describe la
  arborescencia implícita que se recorre en el \emph{backtracking}.}.

Se incluye el campo {\tt num\_clauses} en el tipo {\tt SAT\_status}, para poder
recorrer el arreglo {\tt formula} de todas las cláusulas que componen la fórmula.
\begin{lstlisting}
typedef struct SAT_status{    
    int num_vars;
    int num_clauses;
    clause *formula;
    list *pos_watched_list;
    list *neg_watched_list;
    stack backtracking_status;
    int *model;                     
} SAT_status;
\end{lstlisting}

\subsection{Descripción de la implementación de cláusulas}
En virtud de que en la fase de propagación de restricciones booleanas se optó
por implementar la propagación de cláusulas unitarias con los literales testigos
, cada cláusula exige
dos apuntadores a variables en la misma cláusula. Las variables apuntadas por
los apuntadores {\tt head\_watcher} y {\tt tail\_wacher} son los testigos.

Como cada cláusula es una disjunción de literales, optamos por representarla
como un arreglo de variable llamado {\tt literals} que tiene un tamaño
determinado por el campo {\tt size} de la cláusula. La representación por
arreglo permite recorrer con rapidez una cláusula, operación que habrá que
realizar cada vez que se quiera identificar un literal puro y cada vez que se
necesite actualizar los testigos que permiten reconocer cláusulas unitarias.

\begin{lstlisting}
typedef struct clause{
    int size;
    variable* head_watcher;
    variable* tail_watcher;
    variable* literals;
} clause;
\end{lstlisting}

\subsection{Descripción de la implementación del \emph{Backtracking}}
\subsubsection{\'Arbol implícito del \emph{Backtracking}}\label{backtracking}
Toda implementación de \emph{Backtracking} es un recorrido \emph{Depth-First
  Search} sobre una arborescencia implícita. Esta descripción implícita de la
arborescencia a recorrer exige que se defina cuáles son sus nodos y para cada
nodo, cuáles son sus nodos sucesores. En el caso que nos concierne, los nodos
son de la forma:
\[[x_{i_1}=\mathbb{B},x_{i_2}=\mathbb{B},\ldots, x_{i_k} = \mathbb{B} ], \]
donde $0\leq k \leq n$, con $n$ el número de variables de la instancia del
problema de satisfacción a resolver y $x_{i_k}=\mathbb{B}$ indica que la
variable booleana $x_{i_k}$ tiene un valor booleano (sea $1$ o $0$) asignado. Evidentemente, las
$x_{i_k}$ denotan variables booleanas distintas $\forall k \in \{1,\ldots,n\}$.

Ahora, dado un nodo $[x_{i_1}=\mathbb{B},x_{i_2}=\mathbb{B},\ldots, x_{i_k} =
\mathbb{B} ]$ sus sucesores son todos los nodos de la forma:
$[x_{i_1}=\mathbb{B},x_{i_2}=\mathbb{B},\ldots, x_{i_k} = \mathbb{B},
x_{i_{k+1}}=\mathbb{B} ]$.


\begin{lstlisting}
typedef struct decision_level_data{
    variable assigned_literal;
    int missing_branch;                                                
    list propagated_var;
} decision_level_data;
\end{lstlisting}


\subsection{Problemas encontrados y la manera como fueron resueltos}


\begin{texttt}
d
\end{texttt}

\section{Instrucciones de operación}
Para emplear la aplicación, escribir en la consola el comando
\\ \texttt{sat -f inputfilename -o outputfilename}
\section{Estado Actual}

\section{Conclusiones y recomendaciones}


\end{document}
